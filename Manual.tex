\documentclass[a4paper, 10pt, onecolumn, openright, oneside]{book}
\usepackage[utf8]{inputenc}
% \usepackage[T1]{fontenc}
\usepackage[english]{babel}
\usepackage[top=2cm, bottom=2cm, left=2cm, right=2cm]{geometry}
\usepackage{soul}

\title{RTFM}
\author{Sevla}
\date{\today}
\begin{document}
\maketitle
\pagenumbering{roman}
\tableofcontents

\part{Manual}

\chapter{System}

\section{ls}
\subsection{Synopsis}
\textbf{ls} [\ul{OPTION}]... [\ul{FILE}]...
\subsection{Description}
Liste le contenu d'un répertoire (par ordre alphabétique par défaut)
\subsection{Options}
\begin{description}
\item[-a :] affiche également les fichiers cachés
\item[-F :] affiche les dossiers avec un « / » a la fin et les raccourcis avec un « @ » a la fin
\item[-d :] n'affiche pas le contenu des sous-répertoires
\item[-h :] affiche les tailles d’une manière plus lisible pour l’humain
\item[-t :] trie par date de dernière modification
\item[-t :] affiche sous forme de liste détaillée
\item[-r :] inverse l'ordre d'affichage
\item[-i :] affiche l’inode de chaque fichiers
\end{description}

\section{cd}
\subsection{Synopsis}
\textbf{cd} [\ul{DEST}]
\subsection{Description}
Change de répertoire (par defaut \emph{home})

\section{cp}
\subsection{Synopsis}
\textbf{cp} [\ul{OPTION}]... \ul{SOURCE}... \ul{DIRECTORY}|\ul{FILE}
\subsection{Description}
Copie fichiers et répertoires
\subsection{Options}
\begin{description}
\item[-r :] récursivement
\end{description}

\section{mv}
\subsection{Synopsis}
\textbf{mv} [\ul{OPTION}]... \ul{SOURCE}... \ul{DIRECTORY}|\ul{FILE}
\subsection{Description}
Déplace (et renomme) des fichiers et répertoires
\subsection{Options}

\section{mkdir}
\subsection{Synopsis}
\textbf{mkdir} [\ul{OPTION}]... \ul{DIRECTORY}...
\subsection{Description}
Crée des répertoires
\subsection{Options}
\begin{description}
\item[-p :] crée les répertoires parents si inexistants
\end{description}

\section{rmdir}
\subsection{Synopsis}
\textbf{rmdir} [\ul{OPTION}]... \ul{DIRECTORY}...
\subsection{Description}
Supprime des répertoires (vides)
\subsection{Options}

\section{touch}
\subsection{Synopsis}
\textbf{touch} [\ul{OPTION}]... \ul{FILE}...
\subsection{Description}
Met à jour la date de dernère modification des fichiers (les fichiers sont créé s'ils n'existent pas)
\subsection{Options}

\section{rm}
\subsection{Synopsis}
\textbf{rm} [\ul{OPTION}]... \ul{FILE}...
\subsection{Description}
Supprime des fichiers
\subsection{Options}
\begin{description}
\item[-r :] récursivement
\end{description}


\chapter{Administration}

\chapter{Network}

\part{Keyboard Shortcuts}

\begin{center}
\begin{tabular}{|c|c|}
\hline
\begin{bf}Keyboard Shortcut\end{bf} & \begin{bf}Description\end{bf} \\
\hline
TAB & Auto completion \\
TAB + TAB & List auto-completion possibilities \\
ARROW UP & Previous command \\
ARROW DOWN & Next command \\
CTRL + R & Reverse index search \\
CTRL + L & Clear \\
CTRL + D & End Of File \\
SHIFT + PgUp & Go up in the console messages \\
SHIFT + PgDown & Go down in the console messages \\
CTRL + A $\vert$ HOME & Bring the cursor to the extreme left \\
CTRL + E $\vert$ END & Bring the cursor to the extreme right \\
CTRL + U & Delete everyhing to the left of the cursor \\
CTRL + K & Delete everyhing to the right of the cursor \\
CTRL + W & Delete the fisrt word to the left of the cursor \\
CTRL + Y & Past a word deleted with CTRL+U $\vert$ CTRL+K $\vert$ CTRL+W \\
CTRL + C & Kill a runnig proscessus \\
CTRL + Z & Stop a runnig proscessus \\
\hline
\end{tabular}
\end{center}

\end{document}
