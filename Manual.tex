\documentclass{report}

\usepackage[utf8]{inputenc}
\usepackage[T1]{fontenc}
\usepackage[english]{babel}

\title{RTFM}
\author{Sevla}
\date{July 30, 2019}

\begin{document}

\maketitle

\part{Manual}

\chapter{Administration}

\section{ls}

\subsection{Synopsis}
ls [OPTION]... [FILE]...

\subsection{Description}
liste le contenu d'un répertoire (par ordre alphabétique par défaut)

\subsection{Options}

\subsubsection{-a}
affiche également les fichiers cachés
\subsubsection{-F}
affiche les dossiers avec un « / » a la fin et les raccourcis avec un « @ » a la fin
\subsubsection{-d}
n'affiche pas le contenu des sous-répertoires
\subsubsection{-h}
affiche les tailles d’une manière plus lisible pour l’humain
\subsubsection{-t}
trie par date de dernière modification
\subsubsection{-l}
affiche sous forme de liste détaillée
\subsubsection{-r}
inverse l'ordre d'affichage
\subsubsection{-i}
affiche l’inode de chaque fichiers

\chapter{Network}

\part{Keyboard Shortcuts}

\end{document}
